%!TEX root = thesis.tex

In software engineering, resources such as time, money and developers, are limited. Often when bugs are found in the software developed, bug triaging is used to prioritise bug reports and allocate resources to it. When the number of bugs is considerable, this will require a vast amount of time and effort. The goal of this research is to investigate the usefulness of stack traces in bug reports for the assessment of bug report properties, using existing metrics of bug reports and files, being severity, priority and time-to-fix.

In order to investigate the usefulness of stack traces, a research framework and methodology are developed. Overall, we can conclude that stack traces can be used to link software artifacts. Also, stack traces can be a valuable input for prediction models, for example using metrics of related bugs and source files. 



